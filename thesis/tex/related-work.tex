\chapter{Related work}
\label{cha:related-work}

\section{Homomorphic encryption}
\label{sec:homom-encrypt}
\begin{itemize}
\item Fully Homomorphic Encryption
  \begin{itemize}
  \item theoretically possible
  \item practically not usable
  \end{itemize}
\item Partial Homomorphic Encryption
  \begin{itemize}
  \item partial means with respect to a certain operation
  \item available for multiplication, addition, comparison, ordering, \dots
  \end{itemize}
\end{itemize}

\subsection{CryptDB}
\label{sec:cryptdb}

\cite{CryptDB}
\begin{itemize}
\item use PHE to perform operations on encrypted columns
\item SQL proxy
\item does not support for example average because no operator for division
\end{itemize}

\subsection{MrCrypt}
\label{sec:mrcrypt}

\cite{MrCrypt}
\begin{itemize}
\item written in Scala
\item encryption scheme inference, fails if FHE required due to
  different operators on same value
\end{itemize}

\subsection{Program Analysis for Secure Big Data Processing}
\label{sec:progr-analys-secure}

\cite{ProgramAnalysisBigData}

\begin{itemize}
\item transform pig latin scripts
\item communicate with trusted party to perform re-encryption
\end{itemize}

\subsection{Information-flow control for programming on encrypted data}
\label{sec:inform-flow-contr}

\cite{InfFlowEnc}
\begin{itemize}
\item protect against control flow leaks
\item problem: mix encrypted/plain data, assign to plain data after
  branching with if on encrypted leads to leaks
\end{itemize}

\section{Garbled Circuits}
\label{sec:garbled-circuits}

\begin{itemize}
\item Secure function evaluation
\item generating party vs evaluating party
\end{itemize}

\subsection{Secure Two-Party Computations in ANSI C}
\label{sec:secure-two-party}

\cite{Secure2PartyC}
\begin{itemize}
\item compiler for C
\item use modified CBMC model checker
\end{itemize}

\subsection{VMCrypt}
\label{sec:vmcrypt}

\cite{VMCrypt}
\begin{itemize}
\item library for secure computation using garbled circuits
\item API for integration with external code
\end{itemize}

\subsection{Faster Secure Two-Party Computation Using Garbled Circuits}
\label{sec:faster-secure-two}

\cite{FasterCircuits}
\begin{itemize}
\item instead of sending whole circuit, pipelined approach
\item does not allow global optimizations on circuit
\end{itemize}

\subsection{Reusable Garbled Circuits and Succing Functional Encryption}
\label{sec:reus-garbl-circ}

\cite{ReusableGC}
\begin{itemize}
\item leveled FHE + attribute based encryption + garbled circuits
\item for functions with 1-bit output, larger output by repeating
  scheme (performance? Nothing said\dots)
\end{itemize}

\subsection{Yao's Garbled Circuits: Recent Directions and
  Implementations}
\label{sec:yaos-garbl-circ}

\cite{GCRecent}

\subsubsection{Fairplay-Secure Two-Party Computation System}
\label{sec:fairplay-secure-two}

\cite{Fairplay}

\begin{itemize}
\item loads complete circuit into memory which becomes a problem
\item Requires several hours and 40GB RAM to compile AES
\item Function description language SFDL and compiler
\item high-level programming abstractions inhibit optimizations
\end{itemize}

\section{Functional Programming}
\label{sec:funct-progr}

\subsection{Free Monads}
\label{sec:free-monads}

\cite{Haxl}
- essentially a free monad, here specialized as a DSL for fetching
data from different sources, used by facebook in production

\cite{DataTypesALaCarte}
- at the end presents free monads and offers approach of composing
  languages via the coproduct of functors
- while the composition offers a nice way to use several DSL's not
  necessary here because we have defined all operations the encryption
  schemes support

\subsection{Applicative Functors}
\label{sec:applicative-functors}

Applicatives compose \cite{EssenceIterator}

\subsection{Free Applicative Functors}
\label{sec:free-appl-funct}

- If we resign from the power of monads and use applicative functors,
  we get the ability to analyze the program without executing it
\cite{FreeApplicatives}

%%% Local Variables:
%%% mode: latex
%%% TeX-master: "../thesis"
%%% End:
